\chapter*{Introducere} 
\addcontentsline{toc}{chapter}{Introducere}
Privind în retrospectivă avansul tehnologiei și impactul pe care aceasta îl are asupra noastră in viața de zi cu zi, adaptarea la această nouă realitate nu mai este o opțiune, ci o necesitate.
\par Având acest lucru în vedere, este nevoie acum, mai mult ca niciodată, de introducerea și implementarea a noi soluții tehnologice în cât mai multe domenii și sectoare posibile. Domeniul administrației publice din România este locul perfect pentru îmbrațișarea acestor schimbări, sperând astfel la imbunătațirea eficienței atribuțiilor angajaților.
\par În acest cadru, soluțiile ce vor fi detaliate în documentul ce urmează, constituie un instrument esențial în arsenalul tehnologic necesar, cum este precizat și mai sus, pentru a răspunde provocărilor contemporane.
\par De asemenea, integrarea acestei soluții cu alte sisteme informatice deja prezente și folosite la momentul actual, va permite o comunicare eficientă între diferite agenții și departamente, eliminând astfel din limitările și erorile umane ce sunt prezente.
\section*{Descrierea sumară a soluției}
Proiectul propus are în vedere dezvoltarea unei platforme web, ce oferă diferite functionalități auditorilor publici cât și angajaților departamentelor audidate. Aceștia se vor folosi de platformă pentru a accesa în timp real rezultate atât ale misiunilor de audit care se desfășoara în prezent, cât și ale celor ce au fost efectuate în trecut, resurse care constau în documente si rapoarte oficiale, 
încărcarea și accesarea dovezilor care demonstrează implementarea recomandărilor oferite de auditor, cât și un sistem de notificări în timp real, ce permite utilizatorilor să fie la curent cu cele mai noi informații prezente pe platformă.
\par De asemenea, utilizatorii se vor folosi și de un sistem de exportare a datelor prezente pe site, atât în formate de fișiere .XLS, .CSV sau .DOCS, cât și autocompletarea unor documente oficiale aflate in procedura de audit public de tip șablon cu datele și informațiile pe care auditorul le-a introdus în aplicatie.
\section*{Metodoliga folosită}
Metodologia de lucru la aceast proiect presupune documentarea și analiza procedurilor legislative și a pașilor în desfășurarea unei misiuni de audit public, identificarea problemelor sau a riscurilor ce pot apărea și găsirea soluțiilor în materie de funcționalități prezente în aplicație pentru a le rezolva pe acestea. Dezvoltarea aplicației constă în utilizarea tehnologiei .NET Core:
C\# ASP.NET Core Web API pentru serviciile server-side, ASP.NET Core Blazor pentru serviciile client-side, MySQL pentru stocarea persistentă a datelor respectiv platforma cloud AWS pentru diferite soluții de stocare, S3 Bucket, cât și  servicii de trimitere Email.

\section*{Contribuții}

Cum a fost discutat și în secțiunile anterioare, principalele contribuții pe care le aduce acest proiect asupra procedeului de audit public in România sunt de natură organizatorică, în care utilizatorii, atât auditori cât și reprezentanți ai instituțiilor publice vor putea să se folosească de platformă pentru a executa anumite sarcini ce se dovedesc repetitive.\\
Principalele contribuții aduse de către această soluție sunt:
\begin{itemize}
	
	\item  organizarea misiunilor de audit efectuate, astfel promovând o transparență mărită asupra datelor cât și o eficiență de lucru sporită;
	
	\item facilitarea accesului la resurse partajate între diferitele persoane care participă la misiunea de audit;
	
	\item sistematizarea  si organizarea obiectivelor ce aparțin misiunilor de audit în desfășurare;
	
	\item accesarea unui istoric al misiunilor de audit, astfel auditorul având posibilitatea de a consulta și compara parametrii acestora;
	
	\item stocarea pe platformă a documentelor necesare desfășurării unei misiuni de audit;
	
	\item autocompletarea cu date specificate pe platformă a unor documente oficiale de tip șablon;
	
	\item salvarea datelor de pe platformă în diferite formate cum ar fi : CSV, XLS sau DOCX;
	
	\item sistem de notificări care permite reprezentanților instituțiilor să consulte statusul actual al misiunii de audit și să încarce dovezi referitoare la implementarea recomandărilor;
	
	\item organizarea unei misiuni de audit conform pașilor ce sunt descriși în legislația actuală.
	
\end{itemize}


Lucrarea își propune să ofere în primul capitol o descriere amplă asupra soluției recomandate, discutând problema adresată și explicații asupra punctelor cheie ale soluției, urmată de o prezentare în detaliu referitor la principalele functionalități pe care platforma AudIT le oferă.
\par În capitolul secund, sunt prezentate aspecte ce țin de arhitectura soluției, discutând astfel despre structura \textit{server-ului}, a interfaței grafice, modul de stocare ales cât și de securitatea platformei. De asemenea, sunt discutate și deciziile care au stat la baza alegerilor făcute în materie de tehnologii alese cât și de practici comune in dezvoltarea aplicațiilor  web.
\par  Ultimul capitol prezintă câteva scenarii de utilizare, astfel încat posibilii utilizatori ai acestei platforme să beneficieze de un scurt 'ghid' complet  de utilizare al aplicației.


   

