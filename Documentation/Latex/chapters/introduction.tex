\chapter*{Introducere} 
\addcontentsline{toc}{chapter}{Introducere}
Privind in retrospectiva avansul tehnologiei si impactul pe care aceasta il are asupra noastra in viata de zi cu zi, adaptarea la aceasta noua realitate nu mai este o optiune, ci o necesitate.
\par Avand acest lucru in vedere, este nevoie acum, mai mult ca niciodata, de introducerea si implementarea a noi solutii tehnologice in cat mai multe domenii si sectoare posibile. Domeniul administratiei publice din Romania este locul perfect pentru imbratisarea acestor schimbari, sperand astfel la imbunatatirea eficientei atributiilor angajatilor.
\par In acest cadru, solutiile ce vor fi detaliate in documentul ce urmeaza, constituie un instrument esential 
in arsenalul tehnologic necesar, cum este precizat si mai sus, pentru a raspunde provocarilor contemporane.
\par De asemenea, integrarea acestei solutii cu alte sisteme informatice deja prezente si folosite la momentul actual, va permite o comunicare eficienta intre diferite agentii si departamente, eliminand astfel din limitarile si erorile umane ce sunt prezente.
\section*{Descrierea sumară a soluției}
Proiectul propus are in vedere dezvoltarea unei platforme web, ce ofera diferite functionalitati auditorilor publici cat si angajatilor departamentelor audidate. Acestia se vor folosi de platforma pentru a accesa in timp real rezultate atat ale misiunilor de audit care se desfasoara in prezent, cat si ale celor ce au fost efectuate in trecut, resurse care constau in documente si rapoarte oficiale, 
incarcarea si accesarea dovezilor care demonstreaza implementarea recomandarilor oferite de auditor, cat si un sistem de notificari in timp real, ce permite utilizatorilor sa fie la curent cu cele mai noi informatii prezente pe platforma.
\par De asemenea, utilizatorii se vor folosi si de un sistem de exportare a datelor prezente pe site, atat in formate de fisiere .XLS, .CSV sau .DOCS, cat si autocompletarea unor documente oficiale aflate in procedura de audit public de tip sablon cu datele si informatiile pe care auditorul le-a introdus in aplicatie.
\section*{Metodoliga folosita}
Metodologia de lucru la aceast proiect presupune documentarea si analiza procedurilor legislative si a pasilor in desfasurarea unei misiuni de audit public, identificarea problemelor sau a riscurilor ce pot aparea si gasirea solutiilor in materie de functionalitati prezente in aplicatie pentru a le rezolva pe acestea. Dezvoltarea aplicatiei consta in utilizarea tehnologiei .NET Core:
C\# ASP.NET Core Web API pentru serviciile server-side, ASP.NET Core Blazor pentru serviciile client-side, MySQL pentru stocarea persistenta a datelor respectiv platforma cloud AWS pentru diferite solutii de stocare, S3 Bucket, cat si  servicii de trimitere Email.

\section*{Contributii}

Cum a fost discutat si in sectiunile anterioare, principalele contributii pe care le aduce acest proiect asupra procedeului de audit public in Romania sunt de natura organizatorica, in care utilizatorii, atat auditori cat si reprezentanti ai institutiilor publice vor putea sa se foloseasca de platforma pentru a executa anumite sarcini repetitive.\\
Principalele contributii aduse de catre aceasta solutie sunt:
\begin{itemize}
	
	\item  organizarea misiunilor de audit efectuate, astfel promovand o transparenta marita asupra datelor cat si o eficienta de lucru sporita;
	
	\item facilitarea accesului la resurse partajate intre diferitele persoane care participa la misiunea de audit;
	
	\item sistematizarea  si organizarea obiectivelor ce apartin misiunilor de audit in desfasurare;
	
	\item accesarea unui istoric al misiunilor de audit, astfel auditorul avand posibilitatea de a consulta si compara parametrii acestora;
	
	\item stocarea pe platforma a documentelor necesare desfasurarii unei misiuni de audit;
	
	\item autocompletarea cu date specificate pe platforma a unor documente oficiale de tip sablon;
	
	\item salvarea datelor de pe platforma in diferite formate cum ar fi : CSV, XLS sau DOCX;
	
	\item sistem de notificari care permite reprezentantilor institutiilor sa consulte statusul actual al misiunii de audit si sa incarce dovezi referitoare la implementarea recomandarilor;
	
	\item organizarea unei misiuni de audit conform pasilor ce sunt descrisi in legislatia actuala.
	
\end{itemize}


Lucrarea isi propune sa ofere in primul capitol o descriere ampla asupra solutiei recomandate, discutand problema adresata si explicatii asupra punctelor cheie ale solutiei, urmata de o prezentare in detaliu referitor la principalele functionalitati pe care platforma AudIT le ofera.
\par In capitolul secund, sunt prezentate aspecte ce tin de arhitectura solutiei, discutand astfel despre structura \textit{server-ului}, a interfatei grafice, modul de stocare ales cat si de securitatea platformei. De asemenea, sunt discutate si deciziile care au stat la baza alegerilor facute in materii de tehnologii alese cat si de practici comune in dezvoltarea aplicatiilor  web.
\par  Ultimul capitol prezinta cateva scenarii de utilizare, astfel incat posibilii utilizatori ai acestei platforme sa beneficieze de un scurt 'ghid' complet  de utilizare al aplicatiei.


   

