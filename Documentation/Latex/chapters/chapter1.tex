\chapter{Descrierea aplicatiei AudIT}
Adaptarea la noile paradigme tehnologice ne impune tuturor o provocare,mai mult sau mai putin difica, institutiile publice statului confruntandu-se zilnic cu aceasta problema, este nevoie cat mai repede de o solutie eficienta care va rezolva aceasta problema.\\
Prima sectiune a acestei lucrari urmareste sa exploreze in detaliu cum platforma AudIT se aliniaza si contribuie la actiunea de transformare si adaptare digitala in sectorul public, analizand componentele cheie ale aplicatiei in raport cu problemele pe care aceastea incearca sa le rezolve.


\section{Problema adresata}
Digitalizarea, potrivit definitiei este procesul de transformare a informatiilor dintr-un format analogic, hartii, intr-un format digital,biti. De fapt, acest procedeu constituie o adevarata noua paradigma in materie de algoritmi administrativi, sensul de derulare al intregului sistem si metodele utilizate de catre factorul uman in dezvoltarea solutiilor.\\
In decursul discutiilor  cu tatal meu, auditor public, am descoperit impreumna numeroase 
puncte nevralgice in metodele si solutiile folosite de auditorii publici din Romania pentru a duce la capat anumite intrebuintari de serviciu. Acestea pot parea nesemnificative pe moment, dar observand fenomenul la scara larga, de exemplu pe intreg parcurul unei misiuni de audit public, care poate dura pana la 6 luni, constatam faptul ca intreg procesul si eficienta auditorului sunt major incetinite de aceste imperfectiuni.\\
Una dintre cele mai mari probleme prezente in procesul de audit public, si cel mai probabil in majoritatea institutiilor publice din tara, este nevoia de a folosi si a administra inventarul a multor documente oficiale, pierzand astfel mult timp in identificarea documentului corespunzator actiunii sau activitatii pe care auditorul vrea o sa efectueaze, ulterior pierzand si mai mult timp in completarea si in comunicarea si transmiterea acestui act catre reprezentantul agentiei sau departamentul auditat.De acest lucru este strans legata si problema comunicarii intre partile care participa la misiunea de audit public,aceasta realizandu-se in majoritatea cazurilor prin intermediul postei iar uneori daca distanta permite chiar prin intermediul unor 'curieri umani'. \\
Avand in vedere aceste vulnerabilitati din sistemul public de audit, platforma web AudIT a fost conceputa pentru a raspunde nevoii de adaptare si transformare digitala, incercand in acelasi timp sa imbunatateasca protocoalele si procesele interne, astfel permitand factorului uman sa isi indeplineasca sarcinile intr-un mod mult mai usor si rapid.

\section{Solutia propusa}
Platforma Web AudIT este conceputa ca o solutie inovatoare asupra provocarilor datorate digitalizarii in instituiile publice, oferind un set de instrumente si functionalitati care fac mult mai accesibila si fluenta munca auditorului public cat si cea a reprezentantilor institutiilor audidate.\\
Aplicatie are ca si scop principal cresterea eficientei in procesul de audit public, prin implementarea diferitelor functionalitati care vor imbunatati drastic accesul utilizatorilor la informatii si documente relevante, vor creste nivelul eficientei, auditorii concentrandu-se pe aspecte esentiale ale auditului, fara a-si consuma  astfel timpul si energia pe numeroase sarcini care se pot dovedi repetitive, amanuntite si obositoare in final	respectiv va facilita un mod de comunicare eficace intre persoanele care iau parte la misiunea de audit.



\section{Functionalitatile aplicatiei}
Subsectiunile care vor urma o sa explice in detaliu functionalitatile de baza ale platformei,
modul in care acestea au fost implemnentate, cat si dificultati si provocari ulterioare in ceea ce priveste facilitatile oferite de acestea.

\subsection{Autentificarea pe platforma}
Prima interactiune a fiecarui utilizator cu platforma web o constituie pagina de autentificare, care asigura faptul ca accesul la functionalitatile aplicatiei este restrictionat doar celor care detin sau doresc sa isi creeze un cont pe aceasta aplicatie.\\
Procesul de creare a unui cont nou este conceput astfel incat sa se ajusteze pe necesitatile de securitate de baza ale institutiilor publice.\\
 Presupunand faptul ca fiecare angajat al unui departamant dintr-o institutie a statului detine o adresa de email cu domeniul institutiei de care apartine, tot ce treuie sa faca noul potential utilizator este sa se foloseasca de aceasta adresa de email ca sa isi creeze un cont nou. Contul nou este creat cu drepturi limitate, acesta neavand acces la nici o resursa care apartine de institutia sa pana in momentul cand un reprezentant al institutiei nu ii valideaza contul.\\
 ----INSERT PICTURE HERE---\\
Aceasta metoda de autentificare se bazeaza pe o  configurare initiala a unor utilizatori cu drepturi elevate, reprezentantii departamentelor, carora li se ofera capacitatea de a verifica noii utilizatori care se inregistreaza pe platforma utiliand domeniul departamentului in cauza.Fiind pe o parte un mod in plus prin care se limiteaza accesul utilizatorilor la anumite resurse pana cand identitatea acestora este confirmata, este pe de alta parte un pas necesar care nu prezinta momentan un sistem de automatizare a verificarii identitatii utilizatorilor, eliminand astfel nevoia unei configurari initiale a platformei.

\subsection{Verificarea actiunilor utilizatorilor}
In cadrul aplicatiei, accesul la fiecare entitate este protejata prin implementarea unor liste de acces care definesc permisiunile de scriere si de citire asupra respectivei entitati.Acest lucru se asigura ca initial, fiecare utilizator are drept de scriere si de citire doar asupra resurselor create de acesta pe platforma, ulterior acesta avand posibilitatea de a acorda sau a primi acces de scriere sau citire asupra altor resurse aflate pe platforma.\\
	--INSERT PICTURE HERE---\\
De asemenea, este implementat si un sistem de roluri care restrictioneaza si acestea la randul lor accesul la diferite functionalitati ale aplicatiei, astfel spre exemplu, un utilizator cu rol de reprezentant al unei institutii nu va putea accesa paginile referitoare la crearea sau editarea unei misini de audit.
In plus, pentru o conformitate si pentru o evidenta sporita asupra actiunilor utilizatorilor asupra resurselor de pe platforma este implementat un sistem de auditare al entitatilor, toate operatiile de creare, modificare si stergere fiind salvate.

\subsection{Gestionarea misinilor de audit public}
In cadrul procesului de audit public, o gestioneare eficienta a misiunilor, atat curente cat si din trecut, este esentiala pentru o experienta cat mai naturala si intuitiva a utilizatorului pe platforma.\\
Crearea unei noi misiuni de audit este similara cu crearea unui nou proiect, auditorul specificand numele noii misiuni de audit, institutia respectiv departamenul asupra caruia se realizeaza noua misiune de audit. \\
Dupa crearea noii misiuni, auditorul este redirectionat catre o pagina in care acesta poate vizualiza intr-un tabel toate misiunile de audit la care acesta are acces, cele create de el, dar si cele la care i-a fost oferit accesul.Afisarea intrarilor din tabel este una de tip paginata cu un numar de 7 misiuni pe pagina, astfel incat atentia utilizatorului sa fie concentrata doar pe aceste misiuni, in acest fel eliminand posibilitatea de a nu gasi informatia pe care acesa o cauta datorita unui numar prea mare de linii si informatii.\\
---INSERT PICTURE HERE ---\\
De asemenea, informatiile afisate in acest tabel pot fi sortate alfabetic dupa numele misiunii de audit, dupa starea in care fiecare dintre acestea se afla, dupa departamentul asupra caruia se desfasoara misiunea  sau dupa data ultimei modificari a acesteia.

\subsection{Prelucarea pasilor unei misiuni de audit}
Pentru o reproducere cat mai precisa si cooerenta a stagiilor prin care o misiune de audit trece, platforma permite setarea unui status al fiecarei misiuni de audit, astfel auditorul avand posibilitatea de a-si marca in detaliu progresul pana la momentul curent asupra misiunii de audit.De asemenea, fiecare pas major dintr-o misiune de audit prezinta functionalitati specifice, care vor fi explicate sumar in aceasta subsectiune.

\subsection*{Pregatirea misiunii de audit}
Pregatirea misiunii de audit este etapa initiala in care auditorul creeaza misiunea, consulta misiunile anterioare efectuate la acelasi departament, se elaboreaza un plan de audit, se stabilesc obiectivele, actiunile specifice fiecarui obiectiv respectiv riscurile specifice fiecarei actiuni si se intocmesc o serie de documente oficiale, pentru a tine evidenta activitatilor ulterioare pe care auditorul le va realiza in aceasta misiune de audit.

\subsection*{Interventia la fata locului}
Interventia la fata locului este o etapa importanta a procesului de audit public, etapa care implica de cele mai multe ori o deplasare in teren, auditorul efectueaza interviuri, realizeaza esantioane, analizeaza riscurile si obiectivele stabilite la pasul anterior si incearca sa inteleaga intr-un mod cat mai corect si obiectiv activitatile desfasurate de departamentul respectiv. Acest pas consta in esenta in crearea si completarea a multor documente de tip sablon pe care auditorul le va folosi ulterior in pasii ce urmeaza pentru a intocmi un raport final.

\subsection*{Rezultatele Misiunii}
Dupa finalizarea pasului anterior, auditorul acum dispune de intreg instrumentalul pentru a intocmi un raport final.Acesta este intocmit pe baza diferitelor intalniri intre auditor si repezentantul institutiei in care se discuta aspecte legate de constatarile facute in respectiva misiune de audit. Raportul final cuprinde constatarile facute, recomanandari sub forma 
unor Fise de Identificare si Analiza a Problemei respectiv cauze si consecinte ale problemelor.\\
Acest raport este prezentat partilor particpante la misiune pentru a le informa asupra rezultatelor misiunii de audit si pentru a ajunge la o intelegere asupra termenilor de remediere a problemelor pe care acestia trebuie sa le rezolve.

\subsection*{Urmarirea recomandarilor}
Urmarirea recomandarilor este pasul final dintr-o misine de audit in care sunt monitorizate recomandarile oferite de catre auditor si respectarea termenilor limita de implementare a acestora. Reprezentatii institutiilor trebuie sa ia la cunostinta aceste recomandari si sa gaseasca, ajutati de Fisa de Identificare si Analiza a Problemei corespunzatoare fiecarei recomandari, solutii pentru fiecare chestiune in parte respectand totodata si termenul liminta impus de aceasta.


