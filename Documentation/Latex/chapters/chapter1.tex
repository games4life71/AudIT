\chapter{Descrierea aplicatiei AudIT}
Adaptarea la noile paradigme tehnologice ne impune tuturor o provocare,mai mult sau mai putin difica, institutiile publice statului confruntandu-se zilnic cu aceasta problema, este nevoie cat mai repede de o solutie eficienta care va rezolva aceasta problema.\\
Prima sectiune a acestei lucrari urmareste sa exploreze in detaliu cum platforma AudIT se aliniaza si contribuie la actiunea de transformare si adaptare digitala in sectorul public, analizand componentele cheie ale aplicatiei in raport cu problemele pe care aceastea incearca sa le rezolve.


\section{Problema adresata}
Digitalizarea, potrivit definitiei este procesul de transformare a informatiilor dintr-un format analogic, hartii, intr-un format digital,biti. De fapt, acest procedeu constituie o adevarata noua paradigma in materie de algoritmi administrativi, sensul de derulare al intregului sistem si metodele utilizate de catre factorul uman in dezvoltarea solutiilor.\\
In decursul discutiilor  cu tatal meu, auditor public, am descoperit impreumna numeroase 
puncte nevralgice in metodele si solutiile folosite de auditorii publici din Romania pentru a duce la capat anumite intrebuintari de serviciu. Acestea pot parea nesemnificative pe moment, dar observand fenomenul la scara larga, de exemplu pe intreg parcurul unei misiuni de audit public, care poate dura pana la 6 luni, constatam faptul ca intreg procesul si eficienta auditorului sunt major incetinite de aceste imperfectiuni.\\
Una dintre cele mai mari probleme prezente in procesul de audit public, si cel mai probabil in majoritatea institutiilor publice din tara, este nevoia de a folosi si a administra inventarul a multor documente oficiale, pierzand astfel mult timp in identificarea documentului corespunzator actiunii sau activitatii pe care auditorul vrea o sa efectueaze, ulterior pierzand si mai mult timp in completarea si in comunicarea si transmiterea acestui act catre reprezentantul agentiei sau departamentul auditat.De acest lucru este strans legata si problema comunicarii intre partile care participa la misiunea de audit public,aceasta realizandu-se in majoritatea cazurilor prin intermediul postei iar uneori daca distanta permite chiar prin intermediul unor 'curieri umani'. \\
Avand in vedere aceste vulnerabilitati din sistemul public de audit, platforma web AudIT a fost conceputa pentru a raspunde nevoii de adaptare si transformare digitala, incercand in acelasi timp sa imbunatateasca protocoalele si procesele interne, astfel permitand factorului uman sa isi indeplineasca sarcinile intr-un mod mult mai usor si rapid.

\section{Solutia propusa}
Platforma Web AudIT este conceputa ca o solutie inovatoare asupra provocarilor datorate digitalizarii in instituiile publice, oferind un set de instrumente si functionalitati care fac mult mai accesibila si fluenta munca auditorului public cat si cea a reprezentantilor institutiilor audidate.\\
Aplicatie are ca si scop principal cresterea eficientei in procesul de audit public, prin implementarea diferitelor functionalitati care vor imbunatati drastic accesul utilizatorilor la informatii si documente relevante, vor creste nivelul eficientei, auditorii concentrandu-se pe aspecte esentiale ale auditului, fara a-si consuma  astfel timpul si energia pe numeroase sarcini care se pot dovedi repetitive, amanuntite si obositoare in final	respectiv va facilita un mod de comunicare eficace intre persoanele care iau parte la misiunea de audit.\\



\section{Functionalitatile aplicatiei}

Pellentesque pulvinar pellentesque habitant morbi tristique senectus et. Ornare suspendisse sed nisi lacus sed viverra tellus in hac. Non sodales neque sodales ut etiam sit. In hendrerit gravida rutrum quisque non. Diam quam nulla porttitor massa id neque aliquam. Diam sit amet nisl suscipit adipiscing bibendum est ultricies integer. Cras fermentum odio eu feugiat pretium nibh ipsum. Egestas integer eget aliquet nibh praesent tristique magna. Porttitor eget dolor morbi non arcu risus quis varius quam. Gravida rutrum quisque non tellus orci. Diam volutpat commodo sed egestas egestas.